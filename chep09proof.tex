\documentclass[twoside,floatfix,a4wide]{revtex4}
\usepackage{multirow}
\usepackage{url}
\usepackage{hyperref}
\usepackage{listings}
\usepackage{graphicx}
\usepackage{fancyhdr}
%\usepackage{asymptote}
%\usepackage{amsmath}
\usepackage{color}
\definecolor{mydarkred}{rgb}{0.7,0,0}
\newcommand{\checkme}[1] {\textcolor{mydarkred}{{\bf CHECK ME:}  \{#1\}}}
\newcommand{\comment}[1] {\textcolor{mydarkred}{\em  \{#1\}}}

%\numberwithin{equation}{section} % Eguation numbering with section id. Reguires amsmath -package.
\pagestyle{fancy}
\fancyhead{} % clear all fields

\fancyhead[L]{\bf {\today}} % Left Odd, Right Even

\fancyhead[R]{\thepage}
\fancyfoot[C]{}

\newcommand{\urltilde}[1]{\texttt{#1}} % solves the tilde problem

\hypersetup{
    bookmarks=true,         % show bookmarks bar?
    unicode=false,          % non-Latin characters in Acrobat’s bookmarks
    pdftoolbar=true,        % show Acrobat’s toolbar?
    pdfmenubar=true,        % show Acrobat’s menu?
    pdffitwindow=true,      % page fit to window when opened
    pdftitle={My title},    % title
    pdfauthor={Author},     % author
    pdfsubject={Subject},   % subject of the document
    pdfnewwindow=true,      % links in new window
    pdfkeywords={keywords}, % list of keywords
    colorlinks=true,        % false: boxed links; true: colored links
    linkcolor=red,          % color of internal links
    citecolor=green,        % color of links to bibliography
    filecolor=blue,         % color of file links
    urlcolor=red            % color of external links
}

\begin{document}
\title{Testing PROOF Analysis with Pythia8 Generator Level Data}
\author{\underline{M.~J.~Kortelainen}, A.~Heikkinen, P.~Kaitaniemi, S.~Lehti,
  T.~Lind\'{e}n and L.~Wendland}
\affiliation{Helsinki Institute of Physics, P.O. Box 64, FIN-00014 University of Helsinki, Finland.
  E-mail: matti.kortelainen@cern.ch}

\begin{abstract}
  We study the performance of different ways of running a physics
  analysis in preparation for the analysis of petabytes of data in the
  LHC era. Our test cases include running the analysis code in a Linux
  cluster with a single thread in ROOT, with the Parallel ROOT
  Facility (PROOF), and in parallel via the Grid interface with the
  ARC middleware. We use of the order of millions of Pythia8 generator
  level QCD multi-jet events to stress the analysis system. The
  performances of the test cases are reported.
\end{abstract}

\maketitle
\thispagestyle{fancy}


\appendix
\section{Working notes}

\subsection{Plan}

\begin{itemize}
\item Analysis cases
  \begin{itemize}
  \item Simple analysis (one histogram)
  \item Complex analysis (many histograms, much computing)
  \item TTree output, if feasible
  \end{itemize}
\item Computing cases
  \begin{itemize}
  \item Single thread (ametisti)
  \item PROOF (ametisti)
  \item Grid (ametisti; sepeli if there's time)
  \end{itemize}
\end{itemize}

\subsection{Suggested responsibilities}

\begin{description}
\item[Aatos:]
\item[Lauri:]
\item[Matti:] editor, code
\item[Pekka:]
\item[Sami:]
\item[Tomas:] ametisti
\end{description}

\subsection{Todo}

\begin{itemize}
\item Set up PROOF in ametisti
\end{itemize}

\bibliographystyle{unsrt}
\bibliography{references}

\end{document}
